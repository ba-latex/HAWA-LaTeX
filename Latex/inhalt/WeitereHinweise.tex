\chapter{Weiterführende Bearbeitungshinweise}
\label{weitere-hinweise}
\section{Hinweise zur inhaltlichen Bearbeitung}
\label{weitere-hinweise-inhaltliche-bearbeitung}
\textbf{1. Schritt}
\begin{itemize}
    \item Durchdenken des Themas
    \item Problemstellung definieren
    \item Ziel und Teilziele eindeutig definieren und Bearbeitungsgrenzen festlegen
    \item erste Klärung von Begriffen, Zusammenhängen
    \item Vorgehensweise im Sinne einer Grobgliederung entwickeln
    \item Abstand gewinnen und nochmals überdenken
\end{itemize}
\textbf{2. Schritt}
\begin{itemize}
    \item Sammlung und Erarbeitung des Materials
    \item Sichtung und Zusammenfassung aller in Frage kommenden Quellen
    \item Studium und Bearbeitung der Quellen anhand der Grobgliederung
    \item Vergleich von Lösungsvarianten und Erarbeitung der Lösung
    \item Verfeinerung der Grobgliederung und Konsultation
\end{itemize}
\textbf{3. Schritt}
\begin{itemize}
    \item Textentwurf
    \item endgültige Feingliederung entwerfen
    \item Problemlösung nach definierten Arbeitspaketen (Kapitel)
    \item Auswahl von Abbildungen und Tabellen
    \item Erstellen der Verzeichnisse
\end{itemize}
\textbf{4. Schritt}
\begin{itemize}
    \item Niederschrift und Feinkorrektur
    \item Schreiben der Arbeit \striche{als Ganzes}
    \item Feinkorrektur in sachlicher und formaler Hinsicht
    \item Korrektur durch Dritte und Endredaktion
    \item Ausdruck, Vervielfältigung und Binden
\end{itemize}
\section{Hinweise zum Gebrauch dieser Anleitung}
\label{weitere-hinweise-gebrauch-anleitung}
Diese Richtlinie ist bei der Anfertigung wissenschaftlicher Arbeiten an der Staatlichen Studienakademie anzuwenden.
Sollten Aspekte unberücksichtigt geblieben sein, so sind weitere Medien zu empfehlen:
\begin{itemize}
    \item BECKER, Fred G.: Zitat und Manuskript. Stuttgart, 2007, In: \href{https://www.schaef-fer-poeschel.de/download/zitat/zitat_und_manuskript.pdf}{https://www.schaef-fer-poeschel.de/download/zitat/zitat\_und\_manuskript.pdf} (14.02.2012)
    \item BRINK, Alfred: Anfertigung wissenschaftlicher Arbeiten. 3., überarb. Aufl. Mün-chen, 2007
    \item DECKOW, Frauke: Managementtechniken I, Vorlesungsskript – Wissenschaftli-ches Arbeiten. Glauchau, 2010
    \item HEESEN, Bernd: Wissenschaftliches Arbeiten. 3., durchgesehene und ergänzte Aufl. Berlin, 2014
    \item KORNMEIER, Martin: Wissenschaftlich schreiben leicht gemacht. Bern, 2008
\end{itemize}

Außerdem sind die über den Inhalt dieser Richtlinie hinausgehenden studiengangspezifischen Regelungen zu beachten.

Für Anmerkungen, Hinweise, Korrekturvorschläge zu dieser Anleitung senden Sie eine E-Mail an \href{mailto:richtlinie@ba-glauchau.de}{richtlinie@ba-glauchau.de}.
