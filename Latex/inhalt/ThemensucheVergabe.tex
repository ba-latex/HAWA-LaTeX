\chapter{Themensuche, Themenvergabe}
\label{themensuche-vergabe}
\section{Themensuche}
\label{themensuche}
Die Idee zum Thema der jeweiligen wissenschaftlichen Arbeit kann aus der persönlichen Beobachtung, der Fachpresse und Fachliteratur, aus der Möglichkeit eines Vergleichs oder aus der Spezialisierung der Praxisunternehmen abgeleitet sein.
Für ein erfolgreiches Ergebnis sind das persönliche Interesse und der Bezug zum Thema wichtig.

Das Thema kann theoretischer oder praktischer Art bzw. eine Kombination dieser Möglichkeiten sein.
Ein wissenschaftlicher Neuheitsgrad ist anzustreben.
Für die Projektarbeiten werden begrenzte Problemstellungen in einem Teilbereich des Studiengebietes ausgewählt.
Sie beziehen sich auf praktische Tätigkeiten des aktuellen Theoriesemesters.
Eine Steigerung in der Komplexität und Wissenschaftlichkeit jeder weiteren Projektarbeit ist zu erwarten. (siehe Anhang \ref{anhang-themenvorschlag-projekt})

Für die Bachelorthesis/Diplomarbeit werden durch den Studenten in Abstimmung mit dem Praxispartner dem Studiengangleiter praxisrelevante Themenvorschläge unter Verwendung der zur Verfügung stehenden Formblätter unterbreitet.
Der Themenvorschlag muss auf dem Formblatt (siehe Anhang \ref{anhang-themenvorschlag-bt-dipl}) vom Praxisunternehmen bestätigt werden.
Spezielle Kriterien der Themensuche sind von den Spezifika der einzelnen Studien-gänge abhängig und werden gesondert bekannt gegeben.


\section{Themenvergabe}
\label{themenvergabe}
Über die Themenvergabe entscheidet der Leiter des Studienganges bzw. der Prüfungsausschuss.
Über den Leiter des Studienganges erfolgt entsprechend dem zeitlichen Ablauf die Ausgabe des Themas durch Aushändigung des Themenblattes.
