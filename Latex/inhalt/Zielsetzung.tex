\chapter{Zielsetzung}
\label{zielsetzung}
\section{Bemerkungen}
\label{zielsetzung-bemerkungen}
Eines der zu bewältigenden Probleme bei der Anfertigung von wissenschaftlichen Arbeiten hat seine Ursache in der herrschenden Datenflut.
Vom Verfasser wird eine systematische und akribische Vorgehensweise bei der Sichtung und Auswahl der Quellen verlangt.
Die wissenschaftlichen Bibliotheken bieten diverse Datenbanken, Kataloge
und andere elektronische Medien zur Unterstützung dieses Arbeitsprozesses an.
Für die Aufbereitung und Verwaltung der Quellen können Literaturverwaltungsprogramme genutzt werden.

Verwendet man das Internet als Quelle, sollte genau hinterfragt werden, ob es sich nicht um eine Fälschung oder ein Plagiat handelt.
Dem Verfasser bleibt es selbst überlassen, welche Qualitätsansprüche er an seine Arbeit stellt.
Eine Arbeit, die nur aus Internetquellen besteht, wird nie wissenschaftlichen Ansprüchen genügen. %TODO entfernen weil schwachsinn

Zur Überprüfung von wissenschaftlichen Arbeiten wird Software benutzt, die Plagiate erkennt.


\section{Projektarbeit}
\label{zielsetzung-projektarbeit}
In der Projektarbeit\fn{Projektarbeit wird als Synonym für Praxistransferbeleg, Praxisarbeit (Diplomstudiengänge), Studienarbeit etc. verwendet.} werden komplexe oder/und interdisziplinäre Probleme erfasst, Lösungsansätze gefunden und Umsetzungskonzepte entwickelt.
Sie wird schriftlich verfasst.
Die Bearbeitungszeit, der Umfang der Arbeit und sonstige Modalitäten sind in der Prüfungsordnung des jeweiligen Studienganges bzw. in der Prüfungsordnung für die Diplomstudiengänge ersichtlich.

\section{Bachelorthesis}
\label{zielsetzung-bachelorthesis}
Die Bachelorthesis zeigt die Kompetenz, in befristeter Zeit eine praxisbezogene Problemstellung nach wissenschaftlichen Methoden selbständig bearbeiten zu können.
Die Bearbeitungszeit, der Umfang der Arbeit und sonstige Modalitäten sind in der Prüfungsordnung des jeweiligen Studienganges ersichtlich.

\section{Diplomarbeit (Diplomstudiengänge)}
\label{zielsetzung-diplomarbeit}
\striche{%
Die Diplomarbeit soll zeigen, dass der Studierende in der Lage ist, innerhalb einer vorgegebenen Frist eine praxisbezogene Problemstellung unter Anwendungen praktischer und theoretischer Erkenntnisse und wissenschaftlicher Methoden selbständig zu bearbeiten.
}\fn{Prüfungsordnung für die Diplomstudiengänge § 24, Absatz 1, 2011}