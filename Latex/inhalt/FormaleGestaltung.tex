\chapter{Formale Gestaltung}
\label{formal}
\section{Hinweise zur Anwendung dieser Richtlinie}
\label{formal-hinweise}
Im Folgenden wird dargestellt, wie wissenschaftliche Arbeiten, die schriftliche Prüfungsleistungen im Sinne des Curriculums sind, angefertigt werden und welche formalen Grundsätze dabei Beachtung finden müssen.
Das betrifft insbesondere Projektarbeiten und die Bachelorthesis/Diplomarbeit.
Die Richtlinie wurde nach den dargestellten Methoden verfasst und kann als Beispiel für den Aufbau von wissenschaftlichen Arbeiten angesehen werden.
Für Themen, die in dieser Arbeit nicht erwähnt wurden, findet der Anwender unter Punkt \ref{weitere-hinweise-gebrauch-anleitung} ein Verzeichnis von Medien mit weiterführenden Hinweisen.
\section{Anzahl, Umfang, äußere Gestalt}
\label{formal-anzahl-umfang-gestalt}
Projektarbeiten werden in schriftlicher, ungebundener Ausführung (Hefter, Ringbindung usw.) und zusätzlich in elektronischer Form zum entsprechenden Termin im Sekretariat des Studiengangs abgegeben.
Der Umfang richtet sich nach der Studiengangspezifik und wird schriftlich oder mündlich bekannt gegeben.
Die Bachelorthesis/Diplomarbeit wird in einer gebundenen Ausgabe (Ring- oder Klebebindung) im Sekretariat des Studienganges und in zwei gebundenen Ausgaben, beschriftet mit „Bachelorthesis“ bzw. „Diplomarbeit“ und Name des Autors an die Gutachter übergeben.
Außerdem wird die Arbeit elektronisch in einem \ac{PDF}-Dokument zuzüglich der vollständigen Inhalte der Internet- u. Unveröfflicht-Quellen auf \ac{CD} in die gebundene Arbeit eingeklebt.
Der Umfang richtet sich ebenfalls nach der Studiengangspezifik.
Das Einhalten der Seitenvorgabe ist damit Teil der Aufgabenstellung.
Seitenüber- oder Unterschreitungen werden mit Abzügen bei der Benotung bewertet.
Für die öffentliche Freigabe der Bachelorthesis/Diplomarbeit wird das entsprechende Formular ausgefüllt und vom Praxispartner bestätigt (siehe Anhang 5). %TODO
Es wird nicht in die Arbeit eingebunden, sondern im Sekretariat abgegeben.
Nach der Verteidigung und Benotung der Arbeit werden die Metadaten der zu veröffentlichenden Arbeiten auf dem aktuellen Dokumentenserver der Bibliothek selbständig vom Studierenden eingegeben und das eine \ac{PDF}-Dokument – sofern es öffentlich einsehbar ist – hochgeladen.
Die Anleitung dazu findet man auf der Rückseite des Abmeldeformulars.

\section{Gestaltung der Arbeit}
\label{formal-gestaltung}
\subsection{Seitenangaben}
\label{formal-gestaltung-seitenangaben}
\subsection{Aufbau der Arbeit}
\label{formal-gestaltung-aufbau}
\subsubsection{Prinzipieller formaler Aufbau}
\label{formal-gestaltung-aufbau-prinzipiell}
\subsubsection{Deckblatt}
\label{formal-gestaltung-aufbau-deckblatt}
\subsubsection{Themenblatt}
\label{formal-gestaltung-aufbau-themenblatt}
\subsubsection{Inhaltsverzeichnis}
\label{formal-gestaltung-aufbau-inhaltsverzeichnis}
\subsubsection{Abbildungs- und Tabellenverzeichnis}
\label{formal-gestaltung-aufbau-abbild-tab-verz}
\subsubsection{Abkürzungsverzeichnis}
\label{formal-gestaltung-aufbau-acro-verz}
\subsubsection{Formelverzeichnis}
\label{formal-gestaltung-aufbau-formel-verz}
\subsection{Textteil}
\label{formal-gestaltung-textteil}
\subsubsection{Prinzipieller inhaltlicher Aufbau}
\label{formal-gestaltung-textteil-prinzipiell}
\subsubsection{Gliederung}
\label{formal-gestaltung-textteil-gliederung}
\subsubsection{Layout}
\label{formal-gestaltung-textteil-layout}
\subsubsection{Fußnoten}
\label{formal-gestaltung-textteil-fn}
\subsubsection{Zitierhinweise}
\label{formal-gestaltung-textteil-zitierhinweise}
\subsubsection{Abbildungen und Tabellen im laufenden Text}
\label{formal-gestaltung-textteil-fig-tab-fliesstext}
\subsubsection{Formeln im laufenden Text}
\label{formal-gestaltung-textteil-formeln-fliesstext}
\subsubsection{Sprachliche Anforderungen}
\label{formal-gestaltung-textteil-sprache}
\subsection{Quellenverzeichnis}
\label{formal-gestaltung-quellenverzeichnis}
\subsubsection{Formaler Aufbau und Inhalt}
\label{formal-gestaltung-quellenverzeichnis-inhalt}
\subsubsection{Formaler Aufbau einer Quellenangabe}
\label{formal-gestaltung-quellenverzeichnis-quelle}
\subsection{Anhänge}
\label{formal-gestaltung-anhaenge}
\subsubsection{Anhänge zum Thema der Arbeit}
\label{formal-gestaltung-anhaenge-anhaenge}
\subsubsection{Ehrenwörtliche Erklärung}
\label{formal-gestaltung-anhaenge-erklaerung}
\subsubsection{Abstract zur Bachelorthesis/Diplomarbeit}
\label{formal-gestaltung-anhaenge-abstract}
