\chapter{Formale Gestaltung}
\label{formal}
\section{Hinweise zur Anwendung dieser Richtlinie}
\label{formal-hinweise}
Im Folgenden wird dargestellt, wie wissenschaftliche Arbeiten, die schriftliche Prüfungsleistungen im Sinne des Curriculums sind, angefertigt werden und welche formalen Grundsätze dabei Beachtung finden müssen.
Das betrifft insbesondere Projektarbeiten und die Bachelorthesis/Diplomarbeit.
Die Richtlinie wurde nach den dargestellten Methoden verfasst und kann als Beispiel für den Aufbau von wissenschaftlichen Arbeiten angesehen werden.
Für Themen, die in dieser Arbeit nicht erwähnt wurden, findet der Anwender unter Punkt \ref{weitere-hinweise-gebrauch-anleitung} ein Verzeichnis von Medien mit weiterführenden Hinweisen.
\section{Anzahl, Umfang, äußere Gestalt}
\label{formal-anzahl-umfang-gestalt}
Projektarbeiten werden in schriftlicher, ungebundener Ausführung (Hefter, Ringbindung usw.) und zusätzlich in elektronischer Form zum entsprechenden Termin im Sekretariat des Studiengangs abgegeben.
Der Umfang richtet sich nach der Studiengangspezifik und wird schriftlich oder mündlich bekannt gegeben.
Die Bachelorthesis/Diplomarbeit wird in einer gebundenen Ausgabe (Ring- oder Klebebindung) im Sekretariat des Studienganges und in zwei gebundenen Ausgaben, beschriftet mit „Bachelorthesis“ bzw. „Diplomarbeit“ und Name des Autors an die Gutachter übergeben.
Außerdem wird die Arbeit elektronisch in einem \ac{PDF}-Dokument zuzüglich der vollständigen Inhalte der Internet- u. Unveröfflicht-Quellen auf \ac{CD} in die gebundene Arbeit eingeklebt.
Der Umfang richtet sich ebenfalls nach der Studiengangspezifik.
Das Einhalten der Seitenvorgabe ist damit Teil der Aufgabenstellung.
Seitenüber- oder Unterschreitungen werden mit Abzügen bei der Benotung bewertet.
Für die öffentliche Freigabe der Bachelorthesis/Diplomarbeit wird das entsprechende Formular ausgefüllt und vom Praxispartner bestätigt (siehe Anhang \ref{anhang-fge}).
Es wird nicht in die Arbeit eingebunden, sondern im Sekretariat abgegeben.
Nach der Verteidigung und Benotung der Arbeit werden die Metadaten der zu veröffentlichenden Arbeiten auf dem aktuellen Dokumentenserver der Bibliothek selbständig vom Studierenden eingegeben und das eine \ac{PDF}-Dokument – sofern es öffentlich einsehbar ist – hochgeladen.
Die Anleitung dazu findet man auf der Rückseite des Abmeldeformulars.

\section{Gestaltung der Arbeit}
\label{formal-gestaltung}
\subsection{Seitenangaben}
\label{formal-gestaltung-seitenangaben}
Alle Seiten vor dem eigentlichen Text werden römisch beziffert. (\striche{I, II, …, X, …}).
Das Titelblatt (Deckblatt) wird mitgezählt, erhält aber selbst keine Seitennummer.
Mit Beginn des Textes fängt die arabische Zählweise an (\striche{1, 2, …, 10, …}).
Sie setzt sich bis zur letzten Seite der Arbeit fort (inkl. Anhang- und Quellenverzeichnis).

\subsection{Aufbau der Arbeit}
\label{formal-gestaltung-aufbau}
\subsubsection{Prinzipieller formaler Aufbau}
\label{formal-gestaltung-aufbau-prinzipiell}
Die Arbeiten bestehen grundsätzlich\fn{falls die jeweiligen Teile erforderlich sind} aus folgenden Teilen:
\begin{itemize}
    \item Deckblatt
    \item Sperrvermerk
    \item Themenblatt
    \item Inhaltsverzeichnis
    \item Abbildungsverzeichnis
    \item Tabellenverzeichnis
    \item Formelverzeichnis
    \item Abkürzungsverzeichnis
    \item Textteil (eigentliche Arbeit)
    \item Quellenverzeichnis
    \item Anhangverzeichnis und Anhang
    \item Ehrenwörtliche Erklärung
    \item Abstract (nur bei Bachelorthesis und Diplomarbeit)
\end{itemize}
Jeder dieser Teile beginnt auf einer neuen Seite, ebenso jedes Hauptkapitel der eigentlichen Arbeit.
Bis auf das Abstract sind alle Teile einzubinden.

\subsubsection{Deckblatt}
\label{formal-gestaltung-aufbau-deckblatt}
Das Deckblatt wird wie im Anhang \ref{anhang-deckblatt} angegeben gestaltet.

\subsubsection{Themenblatt}
\label{formal-gestaltung-aufbau-themenblatt}
Das Themenblatt beinhaltet das Original des durch den Prüfungsausschuss bestätigten Themas der Projektarbeit/Bachelorthesis/Diplomarbeit (siehe Anhang \ref{anhang-themenvorschlag-projekt} bzw. Anhang \ref{anhang-themenvorschlag-bt-dipl}).

\subsubsection{Inhaltsverzeichnis}
\label{formal-gestaltung-aufbau-inhaltsverzeichnis}
Das Inhaltsverzeichnis liefert einen Überblick über die bearbeiteten Sachverhalte und dient dem Leser zum schnellen Auffinden der wesentlichen Teile der Arbeit.
Für jeden Eintrag im Inhaltsverzeichnis ist die entsprechende Startseite anzugeben.

Für den Textteil gilt die numerische Gliederung nach DIN 5008 (Beispiel: das Inhaltsverzeichnis dieser Richtlinie)\fn{DIN Deutsches Institut für Normung, 2011}.
Es sollten drei bis vier Gliederungsebenen verwendet werden.
Eintragungen im Inhaltsverzeichnis müssen mit den Kapitelüberschriften im
laufenden Text übereinstimmen.
Hinweise zur Gliederung sind im Punkt \ref{formal-gestaltung-textteil-gliederung} zu finden.

\subsubsection{Abbildungs- und Tabellenverzeichnis}
\label{formal-gestaltung-aufbau-abbild-tab-verz}
Im Abbildungs- und Tabellenverzeichnis sind die Abbildungen und Tabellen mit ihren Titeln und der Seitenangabe fortlaufend numerisch oder kapitelweise sortiert anzuführen.
Abbildungs- und Tabellenverzeichnis können bei Bedarf getrennt werden.
Der formale Aufbau erfolgt analog zum Inhaltsverzeichnis.

\subsubsection{Abkürzungsverzeichnis}
\label{formal-gestaltung-aufbau-acro-verz}
Es sind alle im laufenden Text vorkommenden Abkürzungen alphabetisch geordnet, ohne Angabe von Seitenzahlen aufzuführen und deren Bedeutung anzugeben.
Die im Duden verzeichneten Abkürzungen werden nicht in das  Abkürzungsverzeichnis aufgenommen.

Der formale Aufbau erfolgt analog zum Inhaltsverzeichnis.
Abkürzungen dienen der besseren Lesbarkeit des Textes und nicht dem Autor zur Einsparung von Schreibarbeit, sie müssen im Fachgebiet üblich und eindeutig sein.

\subsubsection{Formelverzeichnis}
\label{formal-gestaltung-aufbau-formel-verz}
Sind Formeln vorhanden, muss ein Verzeichnis angelegt werden.
Es wird fortlaufend numerisch oder kapitelweise mit den zugewiesenen Nummern der Formel und den entsprechenden Seitenangaben aufgeführt.

\subsection{Textteil}
\label{formal-gestaltung-textteil}
\subsubsection{Prinzipieller inhaltlicher Aufbau}
\label{formal-gestaltung-textteil-prinzipiell}
Der Inhalt sollte grundsätzlich aus drei Komplexen bestehen:
\begin{table}
    \begin{tabularx}{\columnwidth}{|l|X|}
        \hline
        \multicolumn{2}{|l|}{\textbf{Inhaltlicher Aufbau}} \\
        \hline
        \textbf{Einleitungsteil:} & Problemstellung – theoretische, praktische, methodische Vorgehensweise, Zielstellung – Abgrenzung der Aufgabenstellung \\
        \hline
        \textbf{Hauptteil:} & Untersuchungen, Analysen, Ergebnisse, Schlussfolgerungen \\
        \hline
        \textbf{Schlussteil:} & Überblick über die Erreichung der Zielstellung, wichtige Ergebnisse, offene Fragen, Weiterführung der Arbeit (Forschungsausblick) \\
        \hline        
    \end{tabularx}
    \caption{Inhaltliche Struktur der Arbeit}
    \label{tab-inhaltl-strukt-arbeit}
\end{table}
\subsubsection{Gliederung}
\label{formal-gestaltung-textteil-gliederung}
Ein Beispiel für den formalen Aufbau der Arbeit bietet die vorliegende Richtlinie.
Durch die Gliederung in Kapitel soll das Thema vollständig untersetzt werden.
Es ist darauf zu achten, dass der "rote Faden" erkennbar ist.
Die Kapitel können weiter untergliedert werden.
Wird ein Haupt- oder Unterpunkt (weiter) untergliedert, müssen mindestens zwei Unterpunkte gebildet werden.
Zwischen den jeweiligen Gliederungsebenen steht kein Text (siehe Abbildung \ref{fig-gliederungsebene} bzw. die Gliederung dieser Arbeit).
Die jeweilige Kapitelüberschrift muss kurz und aussagekräftig sein.

\begin{figure}[H]
    \begin{framed}
        \begin{doublespace}
            \LARGE
            \textbf{%
            3 \quad Formale Gestaltung\\
            \Large
            3.1 \quad Hinweise zur Anwendung dieser Richtlinie\\
            }
            \normalsize
            Text beginnt...
        \end{doublespace}
    \end{framed}
    \caption{Gliederungsebene}
    \label{fig-gliederungsebene}
\end{figure}
\clearpage %! meh, aber in HAWA so
\subsubsection{Layout}
\label{formal-gestaltung-textteil-layout}
Die Tabelle \ref{tab-layout} führt die Vorgaben zur Seitengestaltung und Anordnung aller Seitenele-mente auf.
\begin{table}[H]
    \small
    \setlist[itemize]{nosep, leftmargin=*, before*={\vspace{-.5\baselineskip}}, after*={\vspace{-\baselineskip}}}
    \onehalfspacing
    \begin{tabularx}{\columnwidth}{|p{4cm}|X|}
    \hline
    Papier: & DIN A4, Hochformat, weiß, einseitig beschrieben, Blocksatz\\
    Schriftart: & Arial \\
    Schriftgröße: & Text: 12 Pkt.; Kapitelüberschriften: 12 - 16 Pkt. (siehe diese Richtlinie) \\
    Zeilenabstand: & 1,3-zeilig\\
    Ränder: & oben:\quad 2,0cm\quad\quad\quad unten:\quad2,0cm\\
    & links:\quad2,5cm\quad\quad\quad rechts:\quad2,5cm\\
    Seitennummerierung: & \begin{itemize}
        \item Inhalts-, Abbildungs-, Tabellen- und Abkürzungsverzeichnis mit römi-schen Zahlen (siehe Punkt \ref{formal-gestaltung-seitenangaben})
        \item Textteil mit arabischen Zahlen, beginnend mit 1; Position: unten rechts
        \item einzelne Anhänge mit mehreren Seiten werden in sich arabisch nummeriert
    \end{itemize}\\
    Hauptkapitel: & jedes Hauptkapitel (1. Gliederungsebene) beginnt mit einer neuen Seite\\
    Absätze: & \begin{itemize}
        \item jeder Absatz muss eine inhaltliche Einheit bilden (Gedankengang)
        \item kein Erstzeileneinzug
    \end{itemize}\\
    Hervorhebungen: & \begin{itemize}
        \item müssen einheitlich (\textbf{fett}, \textit{kursiv} und/oder \so{gesperrt}) geschrieben werden
        \item Unterstreichungen vermeiden
    \end{itemize}\\
    Verweis auf Fußnoten: & im laufenden Text hochgestellt und mit kleinerer Schriftart (vgl. \ref{formal-gestaltung-textteil-fn}), Fußnotentext 1-zeilig\\
    Sonstige Hinweise: & \begin{itemize}
        \item Nachnamen von Autoren bzw. Namen von Institutionen im laufenden Text, in Fußnoten und im Quellenverzeichnis in Großbuchstaben
        \item Abkürzungen beim ersten Auftreten erläutern (siehe \ref{formal-gestaltung-aufbau-acro-verz}); in das Abkürzungsverzeichnis aufnehmen
    \end{itemize}\\
    Quellenangaben nach Tabellen, Abbildungen: & 10 Punkt Schriftgröße, 1-zeilig, beginnt unter der Tabellen- bzw. Abbildungsbezeichnung im Tabellen- bzw. Abbildungstitel\\
    Formeln: & mit Formeleditor (bzw. \LaTeX)\\
    Legende zur Formel: & 10 Punkt Schriftgröße, beginnt linksbündig unter der Formel\\
    \hline
    \end{tabularx}
    \caption{Layoutvorgaben}
    \label{tab-layout}
    \singlespacing
    \normalsize
\end{table}
\subsubsection{Fußnoten}
\label{formal-gestaltung-textteil-fn}
Eine Fußnote dient zur Auslagerung von Quellenangaben, Anmerkungen, Legenden, Fachbegriffen, Bemerkungen oder für Erklärungen zum Text oder zur Grafik, um den Textfluss nicht zu beeinträchtigen.
Hinter das betreffende Wort, den Satzteil oder den Satz, dem man eine Anmerkung geben will, wird eine hochgestellte Zahl (\striche{Fußnotenziffer}) angefügt.
Diese Zahl verweist auf eine mit derselben Zahl eingeleitete Stelle im unteren Teil der Seite (Fußnotenteil).
Fußnoten sind fortlaufend arabisch nummeriert am Ende jeder Seite anzuführen.
Sie sind vom laufenden Text durch einen mehrere Zentimeter langen Strich zu trennen, sie sind einzeilig mit kleinerer Schrift (10 Punkt) zu schreiben.\fn{Die üblichen Standardtextverarbeitungssysteme realisieren diese Funktion automatisch.}

Fußnoten, die sich auf eine Tabelle beziehen, können auch unmittelbar nach dieser Tabelle angeführt werden.
Gliederungsüberschriften erhalten keine Fußnoten.

\subsubsection{Zitierhinweise}
\label{formal-gestaltung-textteil-zitierhinweise}
Man unterscheidet das wörtliche vom sinngemäßen Zitieren.
Beim wörtlichen Zitieren werden Anführungszeichen gesetzt.
Der Wortlaut einer Passage wird originalgetreu übernommen.
Beim sinngemäßen Zitieren, dem es auf den Gedankengang ankommt, wird eine Aussage eines Autors mit eigenen Worten wiedergegeben.
Durchgehendes Zitieren ganzer Passagen und zu häufiges wörtliches Zitieren sind zu vermeiden.
Wörtliche Zitate sind zu kennzeichnen und sparsam zu verwenden.
Weiterhin sollen sie nur bei besonders prägnanten Formulierungen verwendet werden und sind im laufenden Text in Anführungszeichen zu setzen.
Auslassungen sind durch eckige Klammern [...] und eigene Ergänzungen durch [Text] im Zitat zu kennzeichnen (siehe Tabelle \ref{tab-woertliches-zitieren}).
\begin{table}[H]
    \begin{tabularx}{\columnwidth}{|p{4cm}|X|}
        \hline
        \multicolumn{2}{|l|}{\textbf{Wörtliches Zitieren}}\\
        \hline\small
        Darstellung im Text & \normalsize In der aktuellen kommunikationspsychologischen Arbeit von Swetlana PHILIPP werden Erstkontakte definiert als \striche{Beschreibung der erstmaligen Begegnung zwischen zwei [oder mehreren] Menschen, die miteinander in Interaktion treten.}$^6$\\
        \hline\small
        Darstellung in der Fußnote & \vspace{.05pt}\normalsize$\overline{^6\,\text{PHILIPP, 2003, S. 39}}$\\
        \hline\small
        Darstellung im Quellenverzeichnis & \normalsize PHILIPP, Swetlana: Kommunikationsstörungen in interkulturellen Erst-Kontakt-Situationen: Dissertation Universität Jena. Jena, 2003\\
        \hline
    \end{tabularx}
    \caption{Wörtliches Zitieren}
    \label{tab-woertliches-zitieren}
\end{table}
Sinngemäße Zitate müssen erkennbar sein.
Das heißt, nicht selbst entwickelte Gedanken, die im laufenden Text dargelegt werden, sind mit einem entsprechenden Quellenverweis zu versehen.
Bezieht sich das sinngemäße Zitieren auf den Satz, setzt man die Fußnote vor dem Punkt.
Für den ganzen Absatz steht die Fußnote nach dem Punkt.
Vor der Originalquelle in der Fußnote wird für das sinngemäße Zitieren \striche{vgl.} gesetzt (siehe Tabelle \ref{tab-sinngem-zitieren}).
Mehrere Quellen einer Fußnote (mehrere Belege eines Gedankens), werden durch Semikolon getrennt.
\begin{table}[H]
    \begin{tabularx}{\columnwidth}{|p{4cm}|X|}
        \hline
        \multicolumn{2}{|l|}{\textbf{Sinngemäßes Zitieren}}\\
        \hline\small
        Darstellung im Text & \normalsize Interkulturen sind Synergiepotenziale des Zusammentreffens von Angehörigen verschiedener Kulturen, welche sich in gemeinsamen kommunikativen Handlungsstrukturen äußert, die den Interagierenden eine gemeinsame Basis für die Kommunikation schafft und diese vorantreibt. Interkulturen verdeutlichen somit auch die Dynamik von Kultur.$^7$\\
        \hline\small
        Darstellung in der Fußnote & \vspace{.05pt}\normalsize$\overline{^7\,\text{vgl. BOLTEN, 2001, S. 70 - 71}}$\\
        \hline\small
        Darstellung im Quellenverzeichnis & \normalsize BOLTEN, Jürgen: Interkulturelle Kompetenz. Erfurt, 2001\\
        \hline
    \end{tabularx}
    \caption{Sinngemäßes Zitieren}
    \label{tab-sinngem-zitieren}
\end{table}

Sekundärzitate sind Zitate, die nicht aus der Originalquelle, sondern aus zweiter Quelle (Sekundärquelle) stammen.
Grundsätzlich ist aus der Originalquelle zu zitieren, da nur so Verfälschungen oder Fehlinterpretationen auszuschließen sind.
Falls die Originalquelle nicht glaubwürdig, nicht auffindbar oder nur verhältnismäßig schwer zugänglich ist, sind ausnahmsweise Sekundärzitate zulässig.
Ein Sekundärzitat ist in der Quellenangabe mit dem Zusatz \striche{zit. nach} zu kennzeichnen.
Im Quellenverzeichnis sind die bibliographischen Angaben von beiden Autoren anzugeben (siehe Tabelle \ref{tab-sekundaerzitate}).
\begin{table}[H]
    \begin{tabularx}{\columnwidth}{|p{4cm}|X|}
        \hline
        \multicolumn{2}{|l|}{\textbf{Sekundärzitate}}\\
        \hline\small
        Darstellung im Text & \normalsize\striche{Ein weiterer Unterschied zwischen Kaizen und Innovation ist darin zu sehen, das Kaizen einer kontinuierlichen Anstrengung und Verpflichtung bei gegebener Technik bedarf, jedoch keiner großen Investition bei der Umsetzung.}$^8$\\
        \hline\small
        Darstellung in der Fußnote & \vspace{.05pt}\normalsize$\overline{^8\,\text{IMAI, 1992, S. 48 - 49 }}\text{zit. nach HARDT, 1998, S. 126}$\\
        \hline\small
        Darstellung im Quellenverzeichnis & \normalsize IMAI, Masaaki: Kaizen – Der Schlüssel zum Erfolg der Japaner im Wettbewerb. München, 1992\\
        & HARDT, Rosemarie: Kostenmanagement – Methoden und Instrumente. München, 1998\\
        \hline
    \end{tabularx}
    \caption{Sekundärzitate}
    \label{tab-sekundaerzitate}
\end{table}

Zitiert man mehrere Veröffentlichungen eines Autors aus demselben Jahr, werden diese durch an die Jahreszahl angehängten Kleinbuchstaben unterschieden (siehe Tabelle \ref{tab-zit-gleich-jahr}).
\begin{table}[H]
    \begin{tabularx}{\columnwidth}{|p{4cm}|X|}
        \hline
        \multicolumn{2}{|l|}{\textbf{Zitieren mit gleichem Erscheinungsjahr}}\\
        \hline\small
        Darstellung im Text & \normalsize\striche{So ist beispielsweise sowohl die Möglichkeit der Anprobe beim
        Fabrikverkauf von Kleidungsstücken als auch die Vermittlung
        von Ehepartnern als Dienstleistung aufzufassen.}$^9$\\
        \hline\small
        Darstellung in der Fußnote & \vspace{.05pt}\normalsize$\overline{^9\,\text{BRUHN, 2006a, S. 19}}$\\
        \hline\small
        Darstellung im Quellenverzeichnis & \normalsize BRUHN, Manfred: Qualitätsmanagement für Dienstleistungen. 6., überarb. Aufl. Berlin, 2006a\\
        & BRUHN, Manfred: Integrierte Kommunikation in deutschsprachigen Ländern. 4., überarb. Aufl. Berlin, 2006b\\
        \hline
    \end{tabularx}
    \caption{Zitieren mit gleichem Erscheinungsjahr}
    \label{tab-zit-gleich-jahr}
\end{table}

Bei Quellen ohne Verfasser, Herausgeber usw. wird der Hauptsachtitel als Ordnungs-wort verwendet (siehe Tabelle \ref{tab-zit-ohne-verfasser}).
\begin{table}[H]
    \begin{tabularx}{\columnwidth}{|p{4cm}|X|}
        \hline
        \multicolumn{2}{|l|}{\textbf{Zitieren ohne Verfasser}}\\
        \hline\small
        Darstellung im Text & \normalsize \striche{Dehnschrauben benutzt man bei dynamischer Beanspru-chung und großen Schaltlängen.}$^{10}$\\
        \hline\small
        Darstellung in der Fußnote & \vspace{.05pt}\normalsize$\overline{^{10}\,\text{Lektor Montagetechnik,}}\text{ 2003, Dehnschraube}$\\
        \hline\small
        Darstellung im Quellenverzeichnis & \normalsize Lektor Montagetechnik: CD-ROM. Berlin, 2003\\
        \hline
    \end{tabularx}
    \caption{Zitieren ohne Verfasser}
    \label{tab-zit-ohne-verfasser}
\end{table}

Zitate von Internet-Quellen sollten sich weitgehend an das Zitieren von Printmedien halten.
Zunächst werden die Autoren aufgeführt (falls vorhanden), anschließend der Titel des Beitrages, dann die \ac{URL}.
Das Datum des Zugriffes ist in der jeweiligen Fußnote auszuweisen.
Sind keine näher bezeichnenden Titel, Überschriften, Seitenzahlen usw. vorhanden, werden die Top-\ac{URL} und die anzuklickenden Links aufgeführt. %TODO was zum fick ist eine Top-URL?!
Außerdem wird eine Kopie des Textinhalts mit Zugriffsdatum auf der \ac{CD} gespeichert, auf der Sie eine digitale (\ac{PDF}-)Version Ihrer Arbeit abgeben, damit diese Quelle auch noch verfolgbar ist, wenn sie im Internet nicht mehr verfügbar sein sollte (siehe Tabelle \ref{tab-zit-online}).
\begin{table}[H]
    \begin{tabularx}{\columnwidth}{|p{4cm}|X|}
        \hline
        \multicolumn{2}{|l|}{\textbf{Zitieren von Internet-Quellen}}\\
        \hline\small
        Darstellung im Text & \normalsize \striche{Die Rechenzentren der Hochschulen bieten im Regelfall jedem Studierenden einen kostenfreien Zugang zum Internet an, zumindest auf dem Campus oder in der Bibliothek.}$^{11}$\\
        & \striche{Medien, die sich nicht im Bestand unserer Bibliothek befinden, können über Leihverkehr bestellt werden. Benutzen Sie dazu bitte dieses Bestellsystem und […].}$^{12}$\\
        \hline\small
        Darstellung in der Fußnote & \vspace{.05pt}\normalsize$\overline{^{11}\,\text{online: BECKER, 2007,}}\text{ S. 9 (14.02.2012)}$
        $^{12}\,\text{online: STAATLICHE STUDIENAKADEMIE GLAUCHAU,}$ 2017 (21.02.2017)\\
        \hline\small
        Darstellung im Quellenverzeichnis & \normalsize BECKER, Fred G.: Zitat und Manuskript. Stuttgart, 2007, In: \href{https://www.schaeffer-poeschel.de/download/zitat/zitat_und_manuskript.pdf}{https://www.schaeffer-poeschel.de/download/zitat/zitat\_und\_manuskript.pdf} (14.02.2012)\\
        & STAATLICHE STUDIENAKADEMIE GLAUCHAU: Fernleihe. In: \href{http://www.ba-glauchau.de/cms/akademie/zentrale-einrichtungen-bibliothek-service.html}{http://www.ba-glauchau.de/cms/akademie/zentrale-einrichtungen-bibliothek-service.html}; Fernleihe (21.02.2017)\\
        \hline
    \end{tabularx}
    \caption{Zitieren von Internet-Quellen}
    \label{tab-zit-online}
\end{table}

Bei Zitaten aus unveröffentlichten Quellen, z.B. unternehmensinternen Dokumenten, Skripten u.ä., hat eine entsprechende Kennzeichnung in Fußnote und Quellenverzeichnis zu erfolgen (siehe Tabelle \ref{tab-unveroeff-quellen}).
\begin{table}[H]
    \begin{tabularx}{\columnwidth}{|p{4cm}|X|}
        \hline
        \multicolumn{2}{|l|}{\textbf{Unveröffentlichte Quellen}}\\
        \hline\small
        Darstellung im Text & \normalsize In diesen Prozess sind die Bereiche Produktion und Vertrieb einzubinden.$^{13}$\\
        \hline\small
        Darstellung in der Fußnote & \vspace{.05pt}\normalsize$\overline{^{13}\,\text{vgl. unveröffentlicht: }}\text{EVAGIS AG, 2017, S. 28}$\\
        \hline\small
        Darstellung im Quellenverzeichnis & \normalsize EVAGIS AG: Arbeitsanweisung Neuprodukteinführungsprozess. Altenburg, 2017 (unveröffentlicht)\\
        \hline
    \end{tabularx}
    \caption{Zitieren aus unveröffentlichten Quellen}
    \label{tab-unveroeff-quellen}
\end{table}
Bei nicht öffentlich zugänglichen Quellen kann bei einmaliger Verwendung die vollständige Quellenangabe auch in der Fußnote erfolgen. Die Belegung des Inhalts erfolgt unabhängig davon wie bei Internetquellen immer in Dateiform auf der \ac{CD}.

Bei Zitierung von Gesetzestexten sind die Vorgaben gemäß Tabelle \ref{tab-zitieren-gesetze} zu beachten.
\begin{table}[H]
    \begin{tabularx}{\columnwidth}{|p{4cm}|X|}
        \hline
        \multicolumn{2}{|l|}{\textbf{Gesetzestextquellen}}\\
        \hline\small
        Darstellung im Text & \normalsize \striche{Kaufmann im Sinne dieses Gesetzbuches ist, wer ein Han-delsgewerbe betreibt.}$^{14}$\\
        \hline\small
        Darstellung in der Fußnote & \vspace{.05pt}\scriptsize$\overline{^{14}\,\text{Handelsgesetzbuch }}\text{(HGB) i. d. F. des Gesetzes vom 25.05.2009, §1 (1)}$\\
        \hline
        Darstellung im Quellenverzeichnis & \normalsize Handelsgesetzbuch (HGB) i. d. F. des Gesetzes vom 25.05.2009. In: Wichtige Gesetze des Wirtschaftsprivatrechts, 11. Aufl. Herne, 2010, S. 23 - 48\\
        \hline
    \end{tabularx}
    \caption{Zitieren aus Gesetzen}
    \label{tab-zitieren-gesetze}
\end{table}

Bei der Zitierung aus Sammelwerken sind die erweiterten Angaben im Quellenverzeichnis zu beachten (siehe Tabelle \ref{tab-sammelwerke}).
\begin{table}[H]
    \begin{tabularx}{\columnwidth}{|p{4cm}|X|}
        \hline
        \multicolumn{2}{|l|}{\textbf{Sammelwerke}}\\
        \hline\small
        Darstellung im Text & \normalsize \striche{Die Einzelgebäude sind meist beliebig auswechselbar.}$^{15}$\\
        \hline\small
        Darstellung in der Fußnote & \vspace{.05pt}\normalsize$\overline{^{15}\,\text{ANGERER; HADLER, }}\text{2005, S. 19}$\\
        \hline\small
        Darstellung im Quellenverzeichnis & \normalsize ANGERER, Fred; HADLER, Gerald: Integration der Verkehrs- in die Stadtplanung. In: STEIERWALD, Gerd; KÜNNE, Hans Dieter; VOGT, Walter [Hrsg.]: Stadtverkehrsplanung. 2., neu bearb. u. erw. Aufl. Berlin, 2005, S. 18 - 28\\
        \hline
    \end{tabularx}
    \caption{Zitieren aus mehrbändigen und fortlaufenden Sammelwerken}
    \label{tab-sammelwerke}
\end{table}

Wird aus fremdsprachigen Quellen zitiert, dann erfolgt bei Verwendung einer eigenen bzw. inoffiziellen Übersetzung die Kennzeichnung als sinngemäßes Zitat.
Nur wenn der Text wie in der Quelle angegeben oder aus einer offiziellen/vom Autor der fremdsprachigen Quelle autorisierten Übersetzung übernommen wird, hat die Kennzeichnung als wörtliches Zitat zu erfolgen.

\subsubsection{Abbildungen und Tabellen im laufenden Text}
\label{formal-gestaltung-textteil-fig-tab-fliesstext}
Tabellen und Abbildungen sind deutlich vom Text zu trennen.
Jede Abbildung und Tabelle erhält eine Bezeichnungs-Unterschrift, die folgendermaßen aufgebaut ist: "Abbildung" bzw. "Tabelle", eine fortlaufende Nummer, Titel der Abbildung bzw. Tabelle.
Die Bezeichnungs-Unterschrift bekommt keinen Punkt.

Die Ausrichtung der Abbildung od. Tabelle erfolgt zentriert, die der Unterschrift linksbündig.

Wird eine Abbildung oder Tabelle einer Publikation unverändert entnommen, dann wird nach dem Titel die Quelle angegeben (siehe Abbildung \ref{fig-schema1}).

\begin{figure}[H]
    \begin{tabularx}{\columnwidth}{ccccccc}
        \cline{1-5}
        \multicolumn{5}{|c|}{Aufwand} & & \\
        \cline{1-5}
        \multicolumn{4}{|c|}{Neutraler Aufwand} & \multicolumn{1}{c|}{Zweckaufwand} & \\
        \multicolumn{1}{|c}{\rotatebox{90}{\parbox{2.5cm}{Sachziel-\\fremder Auf-\\wand}}} & \multicolumn{1}{|c|}{\rotatebox{90}{\parbox{2.5cm}{Perioden-\\fremder Auf-\\wand}}} & 
        \multicolumn{1}{c|}{\rotatebox{90}{\parbox{2.5cm}{Außeror\\-dentlicher\\Aufwand}}} & 
        \multicolumn{1}{c|}{\rotatebox{90}{\parbox{2.5cm}{Bewertungs\\-bedingter\\neutraler\\Aufwand}}} & \multicolumn{1}{c|}{} & & \\
        \cline{1-7}
        & & & & \multicolumn{1}{|c|}{\multirow{2}{*}{Grundkosten}} & Anderskosten & \multicolumn{1}{|c|}{Zusatzkosten} \\
        \cline{6-7}
        & & & & \multicolumn{1}{|c|}{} & \multicolumn{2}{c|}{Kalkulatorische Kosten}\\
        \cline{5-7}
        & & & & \multicolumn{3}{|c|}{Kosten}\\
        \cline{5-7}
    \end{tabularx}
    \caption{Schema 1 zur Überführung der Aufwendungen in Kosten\\(SCHWEITZER; KÜPPER, 2003, S. 18)}
    \label{fig-schema1}
\end{figure}
Wird eine Abbildung oder Tabelle einer Publikation verändert, dann wird nach dem Titel die Quelle entsprechend Abbildung \ref{fig-schema2} angegeben.
\begin{figure}[H]
    \begin{tabularx}{\columnwidth}{cccccc}
        \cline{1-4}
        \multicolumn{4}{|c|}{Aufwendungen in der Finanzbuchhaltung} & & \\
        \cline{1-4}
        \multicolumn{3}{|c|}{neutrale Aufwendungen} & \multicolumn{1}{c|}{\multirow{3}{*}{Zweckaufwand}} & \\
        \multicolumn{1}{|c}{\multirow{2}{*}{\parbox{2cm}{betriebs-\\fremd}}} & \multicolumn{1}{|c}{\multirow{2}{*}{\parbox{2cm}{außer-\\ordentlich}}} & \multicolumn{1}{|c}{\multirow{2}{*}{\parbox{2cm}{perioden-\\fremd}}} & \multicolumn{1}{|c|}{\multirow{2}{*}{}} & & \\
        \multicolumn{1}{|c}{} & \multicolumn{1}{|c}{} & \multicolumn{1}{|c|}{} & \multicolumn{1}{c|}{} & & \\
        \cline{1-6}
        & & & \multicolumn{1}{|c}{} & \multicolumn{1}{|c|}{Anderskosten} &\multicolumn{1}{c|}{} \\
        \cline{5-5}
        & & & \multicolumn{2}{|c|}{Grundkosten} &  \multicolumn{1}{|c|}{Zusatzkosten} \\
        \cline{4-6}
        & & & \multicolumn{3}{|c|}{Kosten in der Kosten- und Leistungsrechnung}\\
        \cline{4-6}
    \end{tabularx}
    \caption{\small Schema 2 zur Überführung der Aufwendungen in Kosten\\(eigene Darstellung in Anlehnung an SCHWEITZER; KÜPPER, 2003, S. 18)\normalsize}
    \label{fig-schema2}
\end{figure}

Abkürzungen, die nur in einer Abbildung oder Tabelle vorkommen, werden in einer Legende erläutert.
Sie erscheinen nicht im Abkürzungsverzeichnis.
Auf jede Abbildung, Tabelle, Formel muss im Text wenigstens einmal verwiesen werden.

\subsubsection{Formeln im laufenden Text}
\label{formal-gestaltung-textteil-formeln-fliesstext}
Formeln bekommen eine zugewiesene Nummer.
Die Formel steht linksbündig.
Die Bezeichnung der Formel wird rechtsbündig angeordnet.
Die einzelnen Formelterme sollten direkt unter der Gleichung in einer Legende erläutert werden, wobei immer die Grundeinheit mit anzugeben ist.
Auf Fachwerte wird direkt verwiesen.

Die Berechnung des Druckverlustes für einen Wasserzähler berechnet sich nach Formel \ref{formel:ohm}.

\formula{$\Delta p_{WZ}=\Delta p\cdot\dfrac{\dot{V}^2_S}{\dot{V}^2_G}$}{%
    $\dot{V}^2_S = $ Spitzendurchfluss $\left[ m^3/h\right]$\\
    $\dot{V}^2_G = $ maximaler Durchfluss im Wasserzähler $\left[ m^3/h\right]$\\
    $\Delta p = $ Druckverlust bei $V_{max} \left[bar\right]$}{Druckverlust}{formel:ohm}


\subsubsection{Sprachliche Anforderungen}
\label{formal-gestaltung-textteil-sprache}
Die Qualität wissenschaftlicher Ausarbeitungen wird nicht nur durch deren Inhalt bestimmt, sondern auch durch Form und Sprache.
Die Beherrschung der vorgeschriebenen Hochsprache\fn{An der Staatlichen Studienakademie Glauchau werden auch Arbeiten in Englischer Sprache gefordert.} wird vorausgesetzt.
Der Schreibstil und die Darstellungsformen von Text und Bild müssen dem fachkundigen Leser das zweifelsfreie Verstehen ermöglichen.

Dazu tragen bei:
\begin{itemize}
    \item eine kritische Wortwahl, eingeschlossen die Vermeidung nicht zwingend notwendiger Fremdwörter (insbesondere von Anglizismen) sowie die Verwendung eindeutiger Fachbegriffe (kein Fachjargon)
    \item die Prüfung aller Aussagen auf ihren Wahrheitsgehalt, eingeschlossen die Unterlassung ungerechtfertigter kausaler Adverbien (deshalb, daher) und nicht nachgewiesener Superlative und Komparative, auch in umschriebener Form wie \striche{das einzig richtige …}
    \item strenge Sachlichkeit, die sich u.a. zeigt in der Vermeidung rhetorischer Füllworte und Floskeln (natürlich, selbstverständlich, also), unkonkreter Angaben (eventuell, früher, heutzutage, in rauen Mengen) und von Konjunktiven in der Ergebnisdarstellung (statt \striche{man sollte...}: \striche{es wird empfohlen…})
    \item ein einfacher Satzbau, eingeschlossen die Vermeidung von Nebensätzen (v.a. geschachtelte), unvollständiger verbaler Aufzählungen und Klammern (erstens, einerseits, zum einen) sowie von Auslassungssätzen (Ellipsen).
\end{itemize}

Rechtschreib- und Grammatikfehler können zur Verfälschung einer Aussage führen.
Rechtschreib- und Grammatikfehler zeigen an, wieweit der Schreiber die vorgeschriebene Hochsprache beherrscht.

Die wissenschaftliche Arbeit wird nicht in der Ich- oder Wir-Form geschrieben, sondern unpersönlich.
\subsection{Quellenverzeichnis}
\label{formal-gestaltung-quellenverzeichnis}
\subsubsection{Formaler Aufbau und Inhalt}
\label{formal-gestaltung-quellenverzeichnis-inhalt}
Das Quellenverzeichnis beginnt auf einer neuen Seite und ist alphabetisch zu ordnen:
\begin{itemize}
    \item nach Autoren/Herausgebern
    \item nach Sachtitel bei Schriften, die keinen Autor/Herausgeber benennen
    \item bei mehreren Schriften eines Autors/Herausgebers, nach dem Sachtitel, danach in zeitlicher Abfolge des Erscheinens
\end{itemize}
Es erfolgt keine Trennung nach Monographien, Aufsätzen, Beiträgen, Sammelwerken, Diplomarbeiten, Dissertationen usw.
Es sind genau die Werke anzuführen, die auch im laufenden Text zitiert worden sind.

\subsubsection{Formaler Aufbau einer Quellenangabe}
\label{formal-gestaltung-quellenverzeichnis-quelle}
Die Angabe von \textbf{Autoren} im Quellenverzeichnis wird anhand der Beispiele \ref{bsp-quelle-autoren} bis \ref{bsp-quelle-mehr-drei-autoren} dargestellt.

\begin{example}[H]
    \begin{framed}
        HOFMANN, Dietrich: Qualitätsmanagement. 3. Aufl. Berlin, 2007
    \end{framed}
    \caption{Quellenangabe Autoren}
    \label{bsp-quelle-autoren}
\end{example}

Bei bis zu drei Autoren pro Quelle werden diese angegeben.

\begin{example}[H]
    \begin{framed}
        HOFMANN, Dietrich; SCHILL, Nikolas; FIEDLER, Rudolf: Elektrotechnik. Berlin, 2005
    \end{framed}
    \caption{Quellenangabe bis zu drei Autoren}
    \label{bsp-quelle-drei-autoren}
\end{example}

Bei mehr als drei Autoren soll der wichtigste Autor (i.d.R. der erstgenannte oder derjenige, der auch Herausgeber ist) genannt werden.
Für alle weiteren Autoren steht \striche{u.a.}.

\begin{example}[H]
    \begin{framed}
        ELSNER, Thomas; u. a.: Bauverträge gestalten. 2., neu bearb. u. erw. Aufl. Bonn, 2003
    \end{framed}
    \caption{Quellenangabe bis zu drei Autoren}
    \label{bsp-quelle-mehr-drei-autoren}
\end{example}

Ist der Autor nicht bekannt, wird der Hauptsachtitel verwendet (siehe Beispiel \ref{bsp-quelle-autor-unbekannt}).

\begin{example}[H]
    \begin{framed}
        Raumkunst. 3., erw. Aufl. Berlin, 2001
    \end{framed}
    \caption{Quellenangabe unbekannter Autor}
    \label{bsp-quelle-autor-unbekannt}
\end{example}

Die Hauptsachtitel sind ungekürzt anzugeben.
Nach dem Titel steht ein Punkt.
Falls für die Kennzeichnung des Werkes ein Zusatz (Diplomarbeit, Dissertation usw.) erforderlich ist, steht nach dem Titel und vor dem Zusatz ein Doppelpunkt mit Leerzeichen (Beispiel \ref{bsp-quelle-diplom-dissertation}).

Folgende Sonderfälle sind zu beachten:\\
Dissertationen und Diplomarbeiten sind als solche als Zusatz zum Sachtitel zu kennzeichnen.
Die Einrichtung, an der sie eingereicht wurden, ist anzugeben (siehe Beispiel \ref{bsp-quelle-diplom-dissertation}).

\begin{example}[H]
    \begin{framed}
        SEEHAUS, Steffen: Konzeptentwurf und Ausführungsplanung der wärmetechnischen Versorgung eines Großkomplexes: Diplomarbeit Studienakademie Glauchau. Glauchau, 2002
    \end{framed}
    \caption{Quellenangabe Dissertationen, Diplomarbeiten}
    \label{bsp-quelle-diplom-dissertation}
\end{example}

Bei mehrbändigen Werken, die nicht fortlaufend seitennummeriert sind und einen eigenen Titel haben, wird die Bandangabe erforderlich (siehe Beispiel \ref{bsp-quelle-mehrbaendig}).

\begin{example}[H]
    \begin{framed}
        BOCHMANN, Fritz: Statik im Bauwesen. Bd. 1: Statisch bestimmte Systeme. 21., bearb. Aufl. Berlin, 2003
    \end{framed}
    \caption{Quellenangabe mehrbändige Werke}
    \label{bsp-quelle-mehrbaendig}
\end{example}

Bei unselbständig erschienenen Schriften wird zur Kennzeichnung "In:" verwendet.
Bei Sammelwerken wird der Autor, Kurztitel angegeben. Nach „In“: führt man das Sammelwerk auf, in dem das zitierte Werk (der Aufsatz) enthalten ist und die Seitenangaben (Anfangsseite - Endseite) des zitierten Werks auf (siehe Beispiel \ref{bsp-quelle-sammelwerk}).

\begin{example}[H]
    \begin{framed}
        ABEL, Paul: Das Insolvenzverfahren. In: CRONE, Andreas [Hrsg.]: Modernes Sanie-rungsmanagement. 2., überarb. Aufl. München, 2010, S. 273 - 308
    \end{framed}
    \caption{Quellenangabe Sammelwerk}
    \label{bsp-quelle-sammelwerk}
\end{example}

Nicht jede Publikation, für die ein Herausgeber genannt wird, ist als Sammelwerk anzusehen und so zu zitieren.
Zuweilen treten Institutionen (bspw. die Deutsche Bundesbank oder Ministerien oder Verbände und Kammern usw.) als Herausgeber auf, ohne dass in der Publikation weitere Autoren einzelner Teile od. Abschnitte genannt werden.
In solchen Fällen ist auf die Kennzeichnung als Herausgeber zu verzichten, da die Institution als Autor auftritt.
In anderen Fällen ist neben der Kennzeichnung einer Person, Personengruppe oder Institution als Herausgeber der konkrete Autor des gesamten Werkes eindeutig ersichtlich.
In beiden Fällen folgt die Zitierweise denen \striche{normaler Publikationen} gemäß der Beispiele \ref{bsp-quelle-autoren} - \ref{bsp-quelle-mehr-drei-autoren}.\fn{Anstelle eines Autors können auch mehrere genannt sein. Entscheidend ist dann, dass alle Autoren den gesamten Inhalt des Werkes gemeinschaftlich verantworten und nicht, wie beim Sammelwerk üblich, nur ausgewählte Teile/Abschnitte.}

Ein Artikel aus einer Zeitschrift oder Zeitung wird in folgender Form angegeben:\\
Verfasser: Titel (des Artikels). In: Titel der Zeitschrift, Jahrgang, Jahr, Heft oder Nr., Seitenangaben der Quelle (siehe Beispiel \ref{bsp-quelle-artikel}).

\begin{example}[H]
    \begin{framed}
        BRUNEY, Chuck: High-Performance Marketing. In: ABA Bank Marketing, Jg. 42, 2010, H. 7, S. 28 - 34
    \end{framed}
    \caption{Quellenangabe Artikel}
    \label{bsp-quelle-artikel}
\end{example}

Die Auflage des Werkes ist anzuführen, sofern es sich nicht um die 1. Auflage handelt.
Die Bezeichnungen können abgekürzt werden.
Nach der Auflage steht ein Punkt (siehe Beispiel \ref{bsp-quelle-auflage}).

\begin{example}[H]
    \begin{framed}
        GALBRAITH, John Kenneth: Der große Crash 1929. 4., völlig überarb. Aufl. München, 2009
    \end{framed}
    \caption{Quellenangabe Auflage}
    \label{bsp-quelle-auflage}
\end{example}

Durch Komma voneinander getrennt werden Verlagsort und Erscheinungsjahr (Jahr der Auflage), sofern dies nicht durch einen der aufgeführten Sonderfälle unnötig ist.
Nur der erste Ort wird angegeben.
Am Ende der Quellenangabe steht kein Punkt, außer nach f. und ff.

Abweichend vom Vorgenannten werden Normendokumente aufgeführt.
Es gilt der Grundsatz: Jeder Buchstabe zählt!
Eine Kurzangabe soll mindestens alle Großbuchstaben, die Dokumentennummer, den Teil (das Blatt, Beiblatt, Supplement, Ergänzung, Änderung o. ä.) und (nach dem Doppelpunkt) das Jahr der Ausgabe enthalten (siehe Beispiel \ref{bsp-quelle-normen}).

\begin{example}[H]
    \begin{framed}
        DIN EN ISO 8503-1:1995\\
        Vorbereitung von Stahloberflächen vor dem Auftragen von Beschichtungsstoffen – Rauheitskenngrößen von gestrahlten Stahloberflächen – Teil 1: Anforderungen und Begriffe für ISO-Rauheitsvergleichsmuster zur Beurteilung gestrahlter Oberflächen (ISO 8503-1:1988); Deutsche Fassung EN ISO 8503-1:1995
    \end{framed}
    \caption{Quellenangabe Normen}
    \label{bsp-quelle-normen}
\end{example}

Bei Internetquellen werden zusätzlich zu den vorhandenen bibliographischen Angaben (Autor/Titel) bzw. Angaben, die die Seite bestmöglich bezeichnen, stets der konkrete Adressname des Dokuments angegeben.
Der vollständige Inhalt der Internetquellen wird auf der \ac{CD} gespeichert, vgl. Punkt \ref{formal-anzahl-umfang-gestalt} (siehe Beispiel \ref{bsp-quelle-internetquellen}).

\begin{example}[H]
    \begin{framed}
        BECKER, Fred G.: Zitat und Manuskript. Stuttgart, 2007, In: \href{https://www.schaeffer-poeschel.de/download/zitat/zitat_und_manuskript.pdf}{https://www.schaeffer-poeschel.de/download/zitat/zitat\_und\_manuskript.pdf} (14.02.2012)
    \end{framed}
    \caption{Quellenangabe Internetquellen}
    \label{bsp-quelle-internetquellen}
\end{example}

Die Zitierweise aller anderer Quellen, die hier nicht einzeln aufgeführt sind (z. B. eigene Forschungs-/Belegarbeiten, unternehmensbezogene/-interne Dokumente, Ergebnisse von Befragungen/Experimenten u.v.a.m.) erfolgt so weit wie möglich an die konkret beschriebenen Fälle angelehnt.
Sofern diese Quellen nicht öffentlich zugänglich sind, sind sie der Arbeit wie Internetquellen beizufügen (auch auszugsweise, in sinn- und zusammenhangwahrender Form) und im Quellenverzeichnis als solche zu kennzeichnen (z.B. durch Nachstellung \striche{(internes Dokument)}, \striche{(unveröffentlichtes Dokument)}, o.s.ä.).

\subsection{Anhänge}
\label{formal-gestaltung-anhaenge}
\subsubsection{Anhänge zum Thema der Arbeit}
\label{formal-gestaltung-anhaenge-anhaenge}
\subsubsection{Ehrenwörtliche Erklärung}
\label{formal-gestaltung-anhaenge-erklaerung}
\subsubsection{Abstract zur Bachelorthesis/Diplomarbeit}
\label{formal-gestaltung-anhaenge-abstract}
